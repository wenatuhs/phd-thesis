%!TEX root = ../main.tex
\chapter{光阴极微波电子枪中的超低束流发射度的基本理论}
\label{chap:theory}

\section{束流相空间与束流发射度}

\section{光阴极微波电子枪中存在的束流发射机制}

\subsection{光电发射}

\subsubsection{单光子光电效应与多光子光电效应}

\subsubsection{表面等离激元加强的光电发射}

\subsection{热发射}

\subsection{场致发射(暗电流)}

\section{光阴极微波电子枪中束流的初始发射度}

\subsection{初始发射度(热发射度)的概念}

\subsection{阴极材料与激光参数对热发射度的影响}

\subsection{阴极表面状况与阴极表面 RF 场对热发射度的影响}

\subsubsection{粗糙度热发射度}

\subsubsection{肖特基效应}

\subsection{暗电流与热发射电流对热发射度的影响}

\subsection{高阶光电流对热发射度的影响}

\section{光阴极微波电子枪中束流发射度的增长}

\subsection{线性力引起的发射度增长}

\subsubsection{切片发射度与投影发射度}

\subsection{非线性力引起的发射度增长}

\subsubsection{非线性 RF 场引起的发射度增长}

\subsubsection{非线性空间电荷力引起的发射度增长}

\subsection{束线元件的球差与色差对束流发射度的影响}

\section{光阴极微波电子枪束流发射度增长的补偿与抑制}

\subsection{线性发射度增长的补偿}

\subsection{非线性发射度增长的抑制}

\subsubsection{KV 分布,盘形束与笔形束}

\paragraph{盘形束纵向膨胀过程中发射度的增长}

\paragraph{笔形束横向膨胀过程中发射度的增长}

\subsubsection{抑制非线性力发射度增长中面临的问题}

\paragraph{非线性 RF 场发射度抑制与非线性空间电荷力发射度抑制之间的矛盾}

\subsection{光阴极注入器的设计与优化}

\section{超低束流发射度的测量}
