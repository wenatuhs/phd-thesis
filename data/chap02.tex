%!TEX root = ../main.tex
\chapter{光阴极微波电子枪中的超低束流发射度的基本理论}
\label{chap:theory}

\section{束流相空间与束流发射度}

\section{光阴极微波电子枪中存在的束流发射机制}

\subsection{光电发射}
光电效应是指光束照射在金属表面会使其发射电子的物理现象,发射出的电子被称为光电子。光电发射具有下面特征:
\begin{itemize}
	\item 对于特定金属而言,入射光频率必须大于某特定值才能产生光电效应,光电效应的发生与否与入射光强度没有关系
	\item 光电效应是一种量子效应,无法用 Maxwell 体系的经典电磁理论解释,必须借助光量子的概念才能解释
	\item 光电效应可用下面著名的公式描述:
		\[
		E = h\nu-\phi
		\]
	其中$E$是出射光电子最大能量,$h\nu$为入射光子能量,$\phi$称为金属的逸出功或功函数(work function)
\end{itemize}

光电效应从发现距今已经 100 多年了,其里程碑有:
\begin{itemize}
  \item 1887 年 H. Hertz 发现用紫外线照射金属电极会更容易产生电火花,当时被称为赫兹效应,这是实验上首次发现光电效应
  \item 1899 年 J. Thomson 实验验证了电子的存在
  \item 1902 年 P. Lenard 实验定性发现出射电子能量与入射光频率正相关,而与入射光强度无关,这与 Maxwell 提出的光的波动理论矛盾
  \item 1905 年 A. Einstein 发文提出光量子概念解释光电效应的实验现象
  \item 1914 年 R. Millikan 实验验证了 Einstein 的光电效应公式
\end{itemize}

\subsubsection{单光子光电效应与多光子光电效应}
\paragraph{单光子光电效应}
单光子光电效应就是金属内电子吸收单个光子的光电效应。

金属内处于费米能级(Fermi Energy)以下的束缚态电子吸收单个光子跃迁到高能态,进而成为自由电子向四周运动;经过若干次散射(能量损失)后部分高能态电子会到达金属表面,其中又有部分电子具有足够能量,克服金属表面势垒后从金属表面逸出,成为光电子。

	1931 年 R. Fowler 研究了光电发射中光电流随入射光子能量变化曲线的拖尾现象。他采用简并电子气的观点,并假设:
	\begin{itemize}
	\item 吸收光子后电子的能量略高于克服表面势垒所需能量,因此忽略电子间散射作用
	\item 所有沿阴极法向的动能与吸收的光子能量之和大于克服阴极表面势垒所需能量的电子都可从表面逃逸,这种电子叫做可出射电子
	\item 光电流正比于可出射电子数
	\end{itemize}
	根据以上假设,R. Fowler 获得了光电发射电流 $I$ 的表达式,成功地揭示了拖尾现象的机理。1933 年 L. Dubridge 扩充了 Fowler 的理论,计算了光电发射中光电子的能谱,因此该理论被称为 Fowler-Dubridge 理论。

我们只演示 Fowler-Dubridge 模型中光电流公式的推导过程。
	假定阴极表面法向沿 $+x$ 方向,利用简并电子气的速度分布公式,可以得到电子密度随 $v_x$ 的分布如下:
	\[
	n(v_x) = \frac{4\pi kT}{m}\left(\frac{m}{h}\right)^2\log\{1+e^{(\mu-\frac{1}{2}mv_x^2)/kT}\}
	\]
	根据 Fowler 的第二条假设,单位体积内所有可出射的电子数为:
	\begin{eqnarray*}
	&N &= \int_{\frac{1}{2}mv_x^2=\phi_w+\mu-h\nu}^{\infty}n(v_x)dv_x\\
	&&= \frac{2\pi kT}{m}\left(\frac{2kT}{m}\right)^{1/2}\left(\frac{m}{h}\right)^3\int_0^{\infty}\frac{\log\{1+e^{-y+(h\nu-\phi_w)/kT}\}}{\{y+(\mu+\phi_w-h\nu)/kT\}^{1/2}}dy
	\end{eqnarray*}
	根据 Fowler 的第一条假设,上式积分分母中的 $y$ 相比于第二项可以忽略,所以上式可简化为:
	\[
	N = \frac{2\sqrt{2}\pi m^{3/2}}{h^3}\frac{(kT)^2}{(\mu+\phi_w-h\nu)^{1/2}}\int_0^{\infty}\log\{1+e^{-y+(h\nu-\phi_w)/kT}\}dy
	\]
	令 $F(x) = \int_0^{\infty}\log(1+e^{-y+x})dy$,则:
	\[
	N = \frac{2\sqrt{2}\pi m^{3/2}}{h^3}\frac{(kT)^2}{(\mu+\phi_w-h\nu)^{1/2}}F\left(\frac{h\nu-\phi_w}{kT}\right)
	\]
	根据 Fowler 的第三条假设,光电流 $I$ 正比于可出射电子数 $N$,我们就得到单光子发射光电流:
	\begin{equation}
	I = CT^2F\left(\frac{h\nu-\phi_w}{kT}\right)
	\end{equation}
	其中 $C$ 为常数,$T$ 为电子气温度,$F(x)$ 为 Fowler 函数,可以展成下面级数:
	\[
	F(x) =
	\begin{cases}
	e^x-\dfrac{e^{2x}}{2^2}+\dfrac{e^{3x}}{3^2}-\cdots & x \le 0\\[10pt]
	\dfrac{\pi^2}{6}+\dfrac{x^2}{2}-\left[e^{-x}-\dfrac{e^{-2x}}{2^2}+\dfrac{e^{-3x}}{3^2}-\cdots\right] & x > 0
	\end{cases}
	\]
	Fowler 函数的特征是在 $x < 0$ 时,其值随 $x$ 的减小而迅速衰减。这个特性意味着波段合适的激光可以作为阴极的发射开关,即光电发射具有良好的可控性。

单光子光电发射的特征如下:
\begin{itemize}
	  \item 光电子数目(光电子产额)与入射光强度成正比,因此又被称为线性光电效应,可以定义光阴极的量子效率 QE 如下:
	  \begin{equation}
	  QE = \frac{N_e}{N_p} = \frac{Q/e}{IS\tau/h\nu} = \frac{J}{I}\cdot\frac{h\nu}{e}
	  \end{equation}
	  其中$N_e$为出射光电子数,$N_p$为入射光子数,$Q$是出射束团电荷量,$J$是出射电流密度,$I$是入射光功率密度,$h\nu$为入射光子能量,$e$为电子电荷量,$S$为入射光横向面积,$\tau$为入射光脉宽
	  \item 入射光子能量必须高于金属的逸出功才可以发生单光子光电效应
\end{itemize}

\paragraph{多光子光电效应}
多光子光电效应就是金属内电子吸收多个光子的光电效应。

金属内处于费米能级以下的束缚态电子同步或非同步吸收多个光子跃迁到高能态,进而成为自由电子向四周运动;经过若干次散射(能量损失)后部分高能态电子会到达金属表面,其中又有部分电子具有足够能量,克服金属表面势垒后从金属表面逸出,成为光电子。

	J. Bechtel 于 1973 年在他的博士论文里推广了 Fowler-Dubridge 模型,将多光子发射的情形包含进来,他得到多光子发射光电流密度:
\begin{equation}
J_n = \alpha_nA\left(\frac{e}{h\nu}\right)^n(1-R)^n\cdot I^nT^2F\left(\frac{nh\nu-\phi_w}{kT}\right)
\end{equation}
其中 $A$ 为 Richardson 常数 \SI{120}{A.cm^{-2}.K^{-2}},$R$ 为阴极表面反射率,$I$ 为入射激光功率密度,$\alpha_n$ 是与 n-光子电离常数成正比的常数。
	
	取 $n=0$,此时 $\alpha_0 = 1$,我们就得到热发射电流密度公式;取 $n=1$,我们就得到单光子光电流密度公式。
	于是总发射电流密度为:
	\begin{equation}
	J = \sum_{i = 0}^{\infty}J_i
	\end{equation}
	此公式包含了热发射电流,线性光电流和高阶光电流。

\begin{itemize}
  \item 光电子产额与入射光强度成幂函数关系,因此又被称为高阶光电效应或非线性光电效应;N光子光电效应的光电流密度J与入射光强度I成N次幂关系:
	  \[
	  J \propto I^N
	  \]
  \item 入射光子能量小于金属逸出功也可能发生多光子光电效应
  \item 入射光照射金属时会同时发生多个高阶光电效应
  \item 高阶光电效应的光电子产额随阶数的增大迅速衰减
\end{itemize}

\subsubsection{表面等离激元加强的光电发射}
	\paragraph{表面等离激元}
	Maxwell 电磁理论预言在两种介电常数实部反号的介质的交界面存在一种沿交界面传播的电磁-电荷密度振荡波,也称做表面等离激元(Surface Plasmon Polaritons, SPPs)。利用经典的 Drude 模型可导出金属的介电常数实部为负,因此金属与介质交界面可以存在 SPPs。
	
	表面等离激元具有下面特点:
	\begin{itemize}
	\item SPPs 存在一个频率上限($\omega_P/\sqrt{2}$),低于此频率的光才能激发 SPPs
	\item 入射光必须设法增加横向动量(横向波数)才能激发 SPPs
	\item SPPs 沿交界面传播,垂直于交界面衰减
	\item 特定波长和角度入射的光可被 SPPs 完全共振吸收,此时交界面处会形成极高电场
	\end{itemize}
	由于 SPPs 有提高入射光吸收率/增强阴极表面光场的性质,可以利用 SPPs 来加强光电发射,提高阴极的量子效率。在金属阴极表面激发起 SPPs,一般采用下面两种方式:
	\begin{itemize}
	\item 利用金属表面的光栅结构来提高入射光的横向波数,这种方式对应 NPC(Nano-Patterned Cathode)阴极
	\item 通过将光射入电介质以提高入射光的横向波数,这种方式对应 ATR(Attenuated Total Reflectance)阴极
	\end{itemize}

\subsection{热发射}
热发射即因金属表面电子热运动而产生的电子发射。

对于温度高于绝对零度的金属,假定其温度为 $T$($T > 0$),由于金属内部的电子能量服从 Fermi-Dirac 分布($T$ 较小时近似 Maxwell-Boltzmann 分布),其拖尾部分的电子处于较高能态,从而具有足够的能量克服金属表面势垒,逸出金属表面。

\paragraph{热发射电流密度}
	阴极上的热发射电流密度可以利用 Drude-Sommerfeld 模型中的电子速度分布公式做近似直接导出。
	
	金属表面的电子速度分布遵守:
	\[n(v_x, v_y, v_z) = 2\left(\frac{m}{h}\right)^3\frac{1}{e^{(E-E_F)/kT}+1}\]
	由于只有动能 $E$ 远高于费米能级 $E_F$ 的电子才能克服金属表面势垒 $U$ 逸出,我们统计热电流时可将 Fermi-Dirac 分布近似为 Maxwell-Boltzmann 分布:
	\[\frac{1}{e^{(E-E_F)/kT}+1} \rightarrow e^{-(E-E_F)/kT}\]
	假定阴极法向沿 $+x$ 方向,那么热发射电流密度 $J$ 可写成:
	\[J = \iiint\limits_{mv_x^2/2 > U}\zskip ev_xn(v_x, v_y, v_z)dv_xdv_ydv_z\]
	将近似后的电子速度分布公式代入,利用高斯积分:
	\[\int_{-\infty}^{\infty}e^{-\alpha z^2}dz=\sqrt{\frac{\pi}{\alpha}}\]
	可以得到:
	\[J = \frac{4\pi emk^2}{h^3}T^2\exp\left(\frac{E_F-U}{kT}\right)\]
	由此我们得到热发射电流密度公式:
\begin{equation}
	J = AT^2\exp\left(-\frac{\phi}{kT}\right)
\end{equation}
	其中 $A=4\pi emk^2/h^3\approx \SI{1202}{\micro A.\milli m^{-2}.K^{-2}}$,称为Richardson常数;$\phi=U-E_F$,定义为金属的逸出功,$k$ 是 Boltzmann 常数。
  \begin{itemize}
	  \item 热发射是光电发射的特例,电子吸收 0 个光子而逸出金属表面,即 0 阶光电发射,电子逃逸是通过热运动完成的
	  \item 热发射电流只与金属温度$T$有关,与入射光强度大小无关
	  \item 金属温度 $T = 0$ 时,热发射电流 $J = 0$;$T > 0$ 时热发射电流 $J > 0$
	  \item 热发射电流的强度很小,在 $T = \SI{300}{K}$ 时热发射电流在 \SI{1}{\micro A} 量级
  \end{itemize}

\subsection{场致发射(暗电流)}
场致发射即因金属表面存在很强的引出电场而产生的电子发射。

当金属表面存在很强的引出电场(即电场线指向金属表面的电场)时,金属表面势垒会在电场的影响下变成准三角形,其厚度变小,从而金属表面的电子可以通过量子隧穿效应穿越表面势垒,逸出金属表面。

在光阴极领域,阴极表面场致发射产生的电流又被称为暗电流。我们下面就尝试推导暗电流密度的解析公式。

考虑阴极表面附近的电子,假定阴极法向沿 $+x$ 方向,利用简并电子气的电子速度分布 $n(v_x, v_y, v_z)$,可写出场致发射电流密度为:
	\[
	J = \iiint\limits_{v_x \in [0, +\infty]}\zskip ev_xn(v_x, v_y, v_z)\cdot \Theta(v_x)dv_xdv_ydv_z
	\]
	其中 $\Theta(v_x)$ 为 $x$ 方向速度为 $v_x$ 的电子的隧穿概率。
	因此想计算场致发射电流密度,需计算电子隧穿概率 $\Theta(v_x)$。
我们采用 WKB 近似(WKB (Wigner, Kramers, Brillouin) approximation)来计算量子隧穿概率。
\paragraph{隧穿概率}
	假定阴极法向沿 $+x$ 方向,我们计算阴极表面势垒分布为 $V(x)$ 情形下的电子波函数,从而得到隧穿概率。设此时电子波函数为 $\Phi(x)$,利用一维含时 Schr\"odinger 方程,我们有:
	\[\frac{d^2\Psi}{dx^2}=\frac{2m(V-E_x)}{\hbar^2}\Psi\]
	其中 $E_x$ 为电子在 $x$ 方向的动能。考虑在 $[x, x+dx]$ 一段的势垒 V,我们近似认为其为常数,那么可解出波函数 $\Phi$ 在此区间满足:
	\[\Phi(x+dx)=\Phi(x)\exp\left(-\frac{\sqrt{2m[V(x)-E_x]}}{\hbar}dx\right)\]
	于是 $\Psi$ 在整个 $x$ 正半轴上的可写为:
	\[\Phi(x)=\Phi(0)\exp\left(-\int_0^x\frac{\sqrt{2m[V(s)-E_x]}}{\hbar}ds\right)\]
	上面的公式就称为 WKB 近似。有了波函数,我们便可以计算特定形状势垒下的隧穿概率:
	\[\Theta(E_x)=\frac{\Psi(L)\Psi^*(L)}{\Psi(0)\Psi^*(0)}=\exp\left(-\frac{2\sqrt{2m}}{\hbar}\int_0^L\zskip\sqrt{V(s)-E_x}ds\right)\]
	式中 $L$ 为势垒宽度,也即相对(能量 $E_x$ 的)势垒高度刚好等于 0 的位置。下面以外场为恒定电场且不考虑空间电荷力时的三角(ET,Exact Triangular)势垒为例计算隧穿概率。
	假定外场梯度为 $F$,那么 $V(x)-E_x=\phi-eFx$,其中 $\phi$ 为相对电子 $x$ 方向动能 $E_x$ 的势垒高度,此时势垒宽度 $L=\phi/eF$。代入 $\Theta$ 的计算公式,就得到 ET 势垒的隧穿概率:
	\begin{equation}
	\Theta(E_x)=\exp\left(-\frac{4\sqrt{2m}}{3e\hbar}\frac{\phi(E_x)^{3/2}}{F}\right)
	\end{equation}
	其中 $\phi(E_x)=\phi_w+E_F-E_x$,是相对于 $x$ 方向动能为 $E_x$ 的电子的金属表面势垒高度。
	
	有了隧穿概率,我们可以代回到场致发射电流密度公式中,计算暗电流密度。

\paragraph{暗电流密度}
	容易将场致发射电流密度公式化为下面形式:
	\[J = A\cdot\frac{T}{k}\int_0^{\infty} dE_x\Theta(E_x)\ln\left(e^{-\frac{E_x-E_F}{kT}}+1\right)\]
	其中 $A$ 为 Richardson 常数。考虑到场致发射电子大部分能量在 Fermi 能级附近,我们可以将一般形式的 $\Theta(E_x)$ 在 $E_F$ 附近做 Taylor 展开到二阶,有:
	\[\Theta(E_x)=\Theta(E_F)\cdot e^{\left(\frac{2\sqrt{2m}}{\hbar}(E_x-E_F)\int\limits_0^L\frac{ds}{\sqrt{V(s)-E_F}}\right)}\]
	将上式代入 $J$ 的表达式,考虑到场致发射电流主要由 $E_x$ 在 Fermi 能级附近的电子贡献,可以将 $J$ 中的积分区间扩充至 $(-\infty, \infty)$。
	通过变量代换及积分公式:
	\[\int_0^{\infty}du u^{-\alpha-1}\ln(u+1)=\frac{\pi}{\alpha\sin(\alpha\pi)},\quad\mathrm{Re}(\alpha)\in(0, 1)\]
	可得到:
	\[J = AT^2\frac{\Theta(E_F)}{C^2\mathrm{Sa}(C\pi)}\]
	其中 $C$ 满足:
	\[C = kT\frac{\sqrt{2m}}{\hbar}\int_0^L\frac{ds}{\sqrt{V(s)-E_F}}\]
	代入 ET 势垒的隧穿概率的相关计算结果,最终我们有场致发射电流密度:
	\begin{equation}
	J=\frac{AF^2}{\phi_w}\exp\left(-\frac{B\phi_w^{3/2}}{F}\right)\frac{1}{\mathrm{Sa}(CT\phi_w^{1/2}/F)}
	\end{equation}
	其中 $A=e^3/8\pi h$,$B=4\sqrt{2m}/3e\hbar$,$C=2\pi k\sqrt{2m}/e\hbar$
	当温度 $T\rightarrow \SI{0}{K}$ 时,$\mathrm{Sa}$ 函数趋近于 1,由此我们有 Fowler-Nordheim 公式:
	\begin{equation}
	J=\frac{AF^2}{\phi_w}\exp\left(-\frac{B\phi_w^{3/2}}{F}\right)
	\end{equation}
	其中 $A=e^3/8\pi h\approx\SI{1.541e-6}{A.eV.V^{-2}}$,$B=4\sqrt{2m}/3e\hbar\approx\SI{6.831}{eV^{-3/2}.V.\nano m^{-1}}$,$F$ 是金属表面场梯度
\begin{itemize}
	\item 场致发射是光电发射的特例,电子吸收 0 个光子而逸出金属表面,即 0 阶光电发射,电子逃逸是通过量子隧穿完成的
	\item 金属温度 $T = \SI{0}{K}$ 时,依然有场致发射电流 $J > 0$
	\item 场致发射电流对金属温度 $T$ 的变化不敏感
	\item 场致发射电流随引出电场的梯度下降而迅速减小,$E = 0$ 时场致发射电流 $J = 0$
\end{itemize}

\section{光阴极微波电子枪中束流的初始发射度}

\subsection{初始发射度(热发射度)的概念}
\subsubsection{三步模型}
	光电子发射的三步模型:
	\begin{itemize}
	\item 光子激发(Excitation)阴极内电子
	\item 电子输运(Transport)到阴极表面
	\item 电子逃逸(Escape)到真空
	\end{itemize}

	Fowler-Dubridge 模型仅仅考虑了阴极中电子在吸收光子后的运动状态,而未考虑光子吸收和电子输运的过程,因此无法用于计算光电子束较细致的参数,例如 QE 和发射度。1958 年 W. Spicer 开发了三步模型,考虑光子吸收和电子输运的过程,成功计算了 QE;2006 年 D. Dowell 利用三步模型,进行了简化并给出了 QE 和发射度的非常简洁的公式,并被实验所验证。
	第一步:光子激发阴极内电子
	假设能量为 $h\nu$ 的光子垂直入射光滑金属阴极(沿 $-z$ 方向入射),由于光子被金属中电子吸收是一个泊松过程,因此光子被吸收的概率会随入射深度 $s$ 呈指数衰减,也即:
	\[
	p(\gamma: \mathrm{surface} \to s) = \frac{1}{\lambda_p}e^{-s/\lambda_p}
	\]
	其中 $p(\gamma: \mathrm{surface} \to s)$ 是进入阴极表面的光子在入射深度 $s$ 处被吸收的概率密度,$\lambda_p$ 是该能量光子在金属中的平均自由程。而能量为 $E$ 的电子吸收该光子,从而从能级 $E$ 跃迁到 $E+h\nu$ 的概率正比于能级 $E$ 的电子数与能级 $E+h\nu$ 的空位数的乘积:
	\[
	p(e^{-}: E\to E+h\nu) \propto f_{\mathrm{FD}}(E)g(E)\cdot [1-f_{\mathrm{FD}}(E+h\nu)]g(E+h\nu)
	\]
	其中 $f_{\mathrm{FD}}(E)$ 为 Fermi-Dirac 分布,$g(E)$ 是能级 $E$ 处的能态密度。
	第二步:电子输运到阴极表面
	被激发到高能态 $E+h\nu$ 的电子开始输运过程。假设被激发的电子运动方向随机,并在 $4\pi$ 方向角均匀分布,再限定被激发的电子能量仅略高于有效逸出功,则电子输运到阴极表面的概率等于电子朝向阴极表面运动并输运过程中无散射的概率(一旦发生散射电子丢失能量,剩余能量不足以克服有效势垒),也即:
	\[
	p(e^{-}: (E+h\nu, s, \theta) \to \mathrm{surface}) = e^{-s/(\lambda_{e-e}(E+h\nu)\cos\theta)}
	\]
	其中 $\lambda_{e-e}(E+h\nu)$ 是能量为 $E+h\nu$ 电子的散射平均自由程,$\theta$ 为电子运动方向与阴极表面法向($+z$)的夹角。
	第三步:电子逃逸到真空
	能量为 $E+h\nu$ 的电子以 $\theta$ 角传输至阴极表面时会尝试克服表面势垒而逃逸。假设所有 $+z$ 方向动能大于等于有效势垒高度(有效逸出功与 Fermi 能量之和)的电子都可以逃逸,可得逃逸概率为:
	\[
	p(e^{-}: \mathrm{escape}) = u[(E+h\nu)\cos^2\theta-\mu-\phi_{\mathrm{eff}}]
	\]
	其中 $u(x)$ 为阶跃函数:$x\ge 0$ 时 $u(x)=1$;$x<0$ 时 $u(x)=0$,$\mu$ 为 Fermi 能量。

\subsubsection{热发射度与量子效率}
利用三步模型,可以计算光阴极的量子效率。
	\begin{equation}
	\mathrm{QE}(\omega) \approx\dfrac{1-R(\omega)}{1+\dfrac{\lambda_{\mathrm{opt}}}{\bar{\lambda}_{e-e}(\omega)}}\frac{(\hbar\omega-\phi_{\mathrm{eff}})^2}{8\phi_{\mathrm{eff}}(E_F+\phi_{\mathrm{eff}})}
	\end{equation}
	
而光阴极的热发射度为:
	\begin{equation}
	\varepsilon_{n} =\sigma\sqrt{\dfrac{\hbar\omega-\phi_{\mathrm{eff}}}{3mc^2}}
	\end{equation}
\subsection{阴极材料与激光参数对热发射度的影响}
\subsubsection{空间电荷限}
	C. Child 于 1911 年提出 Child 定律,该定律是说在平行两极板间的电压 $V$ 固定的情况下,其极板间电流密度存在最大值 $J$,并且 $J$ 与 $V$ 成 3/2 次方关系,因此又被称为 3/2 次方定律。1911 年发表的情况针对的是离子流,I. Langmuir 在 1913 年将 Child 的方法应用于电子流,因此该定律又被称为 Child-Langmuir 定律。

	阴极上空间电荷限的本质是:在两极板间的电流产生的空间电荷场会减小发射极的引出电场,当电流大到一定程度时,发射极表面的总引出电场降到 0,逸出的电子无法获得加速,发射电流无法进一步增加,因此电压固定时发射电流存在饱和值。
	
	Child-Langmuir 模型的假定如下:
	\begin{itemize}
	\item 不考虑相对论效应
	\item 忽略极板间运动的粒子间的散射
	\item 极板间的粒子是同一的(同质量,同电荷量)
	\item 阴极上的初始发射速度为 0
	\end{itemize}
	那么当两极板间电流处于稳态时,我们可以直接利用 Poisson 方程导出空间电荷限电流公式:
	我们考虑两极板间电流中的一点,那么有 Poisson 方程成立:
	\[\nabla^2V=-\frac{\rho}{\epsilon_0}\]
	其中 $V$ 为该点电势,$\rho$ 为该点电荷密度。
	由于阴极上($V=0$)粒子的初始动能为 0,根据能量守恒,电势 $V$ 处粒子的能量满足:
	\[\frac{1}{2}mv^2+qV=0\]
	考虑到电荷守恒及稳态条件,有:
	\[\nabla\cdot\vec{J}=-\frac{\partial \rho}{\partial t}= 0\]
	而电流密度$\vec{J}$满足:
	\[\vec{J}=\rho \vec{v}\]
	我们仅考虑一维情形(束流沿 $z$ 轴运动),可知 $J$ 是个常量,联立以上各式可得关于电势 $V$ 的方程:
	\[\frac{d^2V}{dz^2}+AV^{-\frac{1}{2}}=0\]
	其中 $A=-\frac{J}{\varepsilon_0}\sqrt{-m/2q}$。解上面方程并代入边界条件:
	\begin{eqnarray*}
	&&V|_{z=0}=0\\
	&&V|_{z=d}=V_d
	\end{eqnarray*}
	可得电压$V_d$与最大电流密度 $J$ 之间的关系为:
	\[J = \frac{4\varepsilon_0}{9}\sqrt{-2q/m}\cdot \frac{V_d^{\frac{3}{2}}}{d^2}\]
	对于电子而言,我们就得到阴极上的空间电荷限电流密度为($V_d\rightarrow V$):
	\begin{equation}
	J = \frac{4\varepsilon_0}{9}\sqrt{2e/m}\cdot \frac{V^{\frac{3}{2}}}{d^2}
	\end{equation}
	注意上式我们是在平行电极板情形下推导得到的,但是该公式对于圆柱电极板情形也成立。

	在带有电子枪的电子器件中,人们定义导流系数 $P=I/V^{3/2}$ 来衡量电子枪的发射能力,$P$ 的单位为朴。一般微波器件的电子枪,导流系数在 \SIrange[range-phrase = --]{0.1}{1}{\micro P} 的范围。导流系数反映了阴极发射电流I与阴阳极间电压V之间的关系,它仅取决于枪的几何尺寸。

\subsection{阴极表面状况与阴极表面 RF 场对热发射度的影响}

\subsubsection{粗糙度热发射度}
考虑一个理想二维($x$ 和 $z$)的粗糙表面,表面形态函数 $R(x)$ 为:
\[
	R(x) = a\cos(kx)
\]
那么当条件 $\xi=ak\ll 1$ 满足时,若平均外加电场为 $E$,则阴极表面电场可以如下近似描述:
\begin{eqnarray*}
E_x &=& E\xi\cdot e^{-kz}\sin kx \\
E_z &=& E\left(1+\xi e^{-kz}\cos kx\right)
\end{eqnarray*}
由于刚出射的电子还是非相对论粒子,可以直接用经典力学写出粒子运动方程:
\begin{eqnarray*}
\ddot{x} &=& \frac{eE}{m}\xi\cdot e^{-kz}\sin kx \\
\ddot{z} &=& \frac{eE}{m}\left(1+\xi e^{-kz}\cos kx\right)
\end{eqnarray*}
为了推导公式的简洁,需做几个较大的近似:
\begin{itemize}
\item 认为电子出射方向近似是 $z$ 向
\item 极短的运动过程中横向位置 $x$ 基本不发生改变
\end{itemize}
令 $eE/m=A$,积分上面第一式,有:
\[
	\dot{x}_{\infty} - \dot{x}_0 \approx A\xi\sin kx\int_0^{\infty}\zskip e^{-kz}\,dt
\]
考虑到 $\xi \ll 1$,$z$ 向运动方程可以近似为:
\[
\ddot{z}\approx A
\]
因此 $z$ 和 $t$ 的关系就可以直接积分得到:
\[
z\approx \dot{z}_0 t + \dfrac{1}{2}At^2
\]
将上式代入横向速度方程,就得到:
\[
	\dot{x}_{\infty} - \dot{x}_0 \approx A\xi\sin kx\int_0^{\infty}\zskip e^{-k\left(\dot{z}_0 t + \frac{1}{2}At^2\right)}\,dt
\]
由于电子初始纵向速度 $\dot{z}_0$ 很小,忽略指数上时间的线性项,直接由高斯分布积分公式得到:
\[
	\dot{x}_{\infty} - \dot{x}_0 \approx \sqrt{\frac{\pi A}{2k}}\cdot\xi\sin kx
\]
因此归一化散角 $\Delta_{\infty}$ 就满足:
\begin{eqnarray*}
\Delta_{\infty}^2 &=& \frac{\left\langle\dot{x}_{\infty}^2\right\rangle}{c^2} \approx \frac{\left\langle\left(\dot{x}_0 + \sqrt{\frac{\pi A}{2k}}\cdot\xi\sin kx\right)^2\right\rangle}{c^2} \\
	&=& \frac{\left\langle\dot{x}_{0}^2\right\rangle}{c^2} + 2\sqrt{\frac{\pi A}{2k}}\xi\frac{\left\langle\sin kx\cdot\dot{x}_{0}\right\rangle}{c^2} + \frac{\pi A}{2k}\xi^2\frac{\left\langle\sin^2 kx\right\rangle}{c^2}
\end{eqnarray*}
若假定交叉项为 0,那么上式可化为:
\[
	\Delta_{\infty}^2 \approx \Delta_{0}^2 + \frac{\pi A\xi^2}{4kc^2}
\]
将 $A,\xi$ 的表达式代回上式,就得到:
\[
	\Delta_{\infty}^2 \approx \Delta_{0}^2 + \frac{\pi e}{4mc^2}\cdot a^2kE
\]
上式中,约等号右边第二项便是杂散电场对热发射度的贡献,其大小与外加电场强度成正比。比较杂散电场贡献与原随机贡献的大小,有:
\[
	\eta = \frac{\pi e}{4mc^2}\cdot a^2kE/\Delta_{0}^2 = \dfrac{3}{4}\pi e\dfrac{a^2kE}{\hbar\omega-\phi_{\text{eff}}}
\]
取 Dowell 文章中提供的参数以及何小中博士论文中的宏观粗糙度参数:
\[
	a = 100\,\text{nm},\quad
	\lambda = 16\,\mu\text{m}
\]
可得:
\[
	\eta = 0.57
\]
可见杂散电场使发射度增长了 $\sqrt{\eta+1}-1\approx 25\%$,这表明杂散电场是导致热发射度增长的一个主要原因。

\subsubsection{肖特基效应}
	电子在逃逸出金属阴极表面时要克服表面势垒(真空势垒和镜像电荷势的总和)的作用;阴极表面的高引出场强会降低表面势垒的高度,从而降低电子所感受到的逸出功,这种效应叫做 Schottky 效应,电子所感受的逸出功叫做有效逸出功。

	对于单个电子而言,在金属阴极表面距离 $x$ 的位置,其电势能为:
	\[
	\Phi = \phi_{\mathrm{w}} - \frac{e^2}{16\pi\varepsilon_0x} - eE_0x
	\]
	其中 $\phi_{\mathrm{w}}$ 为金属的逸出功,$-e^2/16\pi\varepsilon_0x$ 是镜像电荷势能,$- eE_0x$ 是引出场势能。
	容易看出总势能在距离 $x$ 满足:
	\[
	x = \sqrt{\frac{e}{16\pi\varepsilon_0E_0}}
	\]
	时取得最大值,最大值即为有效逸出功:
	\begin{equation}
	\phi_{\mathrm{eff}} = \phi_{\mathrm{w}} - e\sqrt{\frac{eE_0}{4\pi\varepsilon_0}}
	\end{equation}

\subsection{暗电流与热发射电流对热发射度的影响}

\subsection{高阶光电流对热发射度的影响}

\section{光阴极微波电子枪中束流发射度的增长}

\subsection{线性力引起的发射度增长}

\subsubsection{切片发射度与投影发射度}

\subsection{非线性力引起的发射度增长}

\subsubsection{非线性 RF 场引起的发射度增长}

\subsubsection{非线性空间电荷力引起的发射度增长}

\subsection{束线元件的球差与色差对束流发射度的影响}

\section{光阴极微波电子枪束流发射度增长的补偿与抑制}

\subsection{线性发射度增长的补偿}

\subsection{非线性发射度增长的抑制}

\subsubsection{KV 分布,盘形束与笔形束}

\paragraph{盘形束纵向膨胀过程中发射度的增长}

\paragraph{笔形束横向膨胀过程中发射度的增长}

\subsubsection{抑制非线性力发射度增长中面临的问题}

\paragraph{非线性 RF 场发射度抑制与非线性空间电荷力发射度抑制之间的矛盾}

\subsection{光阴极注入器的设计与优化}

\section{超低束流发射度的测量}
