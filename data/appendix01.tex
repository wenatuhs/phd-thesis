%!TEX root = ../main.tex
\chapter{其它附录}
\section{\label{ap:gpsf}光滑光阴极上光电子动量的一阶矩和二阶矩}
章节 \ref{chap:roughness} 中涉及许多对光电子动量的统计,这里给出光滑光阴极正入射情形下光电子动量一阶矩和二阶矩的结果。

光滑光阴极正入射情形下光电子的相空间分布为:
\begin{equation}
D(x,y,p_x,p_y,p_z) = C\cdot\dfrac{p_z}{\sqrt{p_z^2+p_m^2}\cdot\sqrt{p_x^2+p_y^2+p_z^2+p_m^2}}
\label{eq:apbd}
\end{equation}
方便起见,令 $\Delta=p_M-p_m$ 以及 $\delta=(p_M-p_m)/p_m$,采用式 \ref{eq:apbd} 中的光电子分布,可以统计出各方向动量一阶矩和二阶矩为:
\begin{eqnarray}
\left\langle p_x\right\rangle &=& 0 \nonumber\\
\left\langle p_y\right\rangle &=& 0 \nonumber\\
\left\langle p_z\right\rangle & \approx & p_m\left[\frac{8\sqrt{2}}{15}\sqrt{\delta}+\frac{2\sqrt{2}}{35}\left(\sqrt{\delta}\right)^3\right]\nonumber\\
\left\langle p_x^2\right\rangle &=& \frac{1}{6} \Delta \left(\Delta + 2 p_{m}\right) \nonumber\\
\left\langle p_y^2\right\rangle &=& \frac{1}{6} \Delta \left(\Delta + 2 p_{m}\right) \nonumber\\
\left\langle p_z^2\right\rangle &=& \frac{1}{6} \Delta \left(\Delta + 4 p_{m}\right)
\end{eqnarray}
注意上面各式的成立条件是体发射,金属光阴极,光滑表面及正入射。

\section{\label{ap:stat}公式 \ref{eq:stat} 的推导}
为了看清 \ref{eq:stat} 的来源且不失一般性,将 3D 情形简化为 2D 情形,于是 $v(x, y, p_x, p_y, p_z)$ 退化为 $v(x, p_x)$,激光的功率密度分布退化为 $I(x)$,点扩散函数退化为 $f(x, p_x)$。

下面直接对 $v$ 在 $D$ 下进行统计。
\begin{eqnarray*}
\langle v \rangle_{D} &=& \langle v\rangle_{I * f} = \iint dxdp_xv(x, p_x)\cdot I*f(x, p_x)\\
&=& \iint dxdp_xv(x, p_x)\int dx^{\prime}I(x^{\prime})\cdot f(x-x^{\prime}, p_x)\\
&&\text{Let } x_I = x^{\prime}\text{, } x_f = x-x^{\prime} \\
&=& \iint dx_fdp_xf(x_f, p_x)\int dx_II(x_I)\cdot v(x_I+x_f, p_x)\\
&=& \langle\langle v(x_I+x_f, p_x)\rangle_{I}\rangle_{f} = \langle\langle v(x_I+x_f, p_x)\rangle_{f}\rangle_{I}\\[6pt]
&=& \langle v(x_I+x_f, p_x)\rangle_{I\cdot f}
\end{eqnarray*}
注意到 $(x, x^{\prime})$ 到 $(x_f, x_I)$ 的雅可比行列式为 1,从而有 $dxdx^{\prime}=dx_Idx_f$,这就得到了式 \ref{eq:stat}。
